\documentclass[12pt,a4paper]{article}

% --------------------- Packages ---------------------
\usepackage[utf8]{inputenc}
\usepackage[T1]{fontenc}
\usepackage[polish]{babel}
\usepackage{lmodern}
\usepackage[margin=2.5cm]{geometry}
\usepackage{graphicx}
\usepackage{amsmath, amssymb}
\usepackage{booktabs}
\usepackage{caption}
\usepackage{hyperref}
\usepackage{float}
\usepackage{titlesec}
\usepackage{xcolor}

% --------------------- Formatting ---------------------
\titleformat{\section}{\large\bfseries\sffamily}{\thesection}{1em}{}
\titleformat{\subsection}{\normalsize\bfseries\sffamily}{\thesubsection}{1em}{}
\renewcommand{\familydefault}{\sfdefault}

% --------------------- Title ---------------------
\title{\textbf{AgPMS}\\ \large \emph{Silver nanowire PVB melamine sponge electrodes for EEG}}
\author{KN Neuroinformatyki \& "Nanorurki" \\
\small Uniwersytet Warszawski, Wydział Fizyki}
\date{\today}

% --------------------- Document ---------------------
\begin{document}
\maketitle
\hrule
\vspace{1em}

% --------------------- Abstract ---------------------
\begin{abstract}
    \noindent
    Celem projektu jest odtworzenie i ocena skuteczności nowatorskich elektrod EEG wykonanych z elastycznej gąbki melaminowej pokrytej nanodrutami srebra. Elektrody te mają zastąpić tradycyjne elektrody żelowe w nieinwazyjnych interfejsach mózg–komputer (BCI), oferując lepszy kontakt ze skórą bez potrzeby stosowania żelu przewodzącego. Zakładamy wytworzenie elektrod o wysokiej przewodności, stabilności mechanicznej oraz zdolności do rejestracji sygnałów EEG porównywalnej z klasycznymi rozwiązaniami. Projekt ma na celu potwierdzenie parametrów technicznych i funkcjonalnych opisanych w literaturze oraz ich potencjalne zastosowanie w praktycznych systemach BCI.

\end{abstract}

\vspace{1em}
\hrule
\vspace{1.5em}

% --------------------- Sections ---------------------

\section{Wprowadzenie}
Brief overview of the theoretical background and relevance of the work. State the objective of the project clearly. If applicable, define a hypothesis or scientific question.

\section{Niezbędne materiały i sprzęt}

\subsection{Equipment}
\begin{table}[H]
    \centering
    \caption{List of Equipment}
    \begin{tabular}{@{}llll@{}}
        \toprule
        \textbf{Device} & \textbf{Model}  & \textbf{Manufacturer} & \textbf{Function}       \\
        \midrule
        Spin Coater     & KW-4A           & Chemat                & Thin film deposition    \\
        Oven            & Memmert ULE 400 & Memmert               & Thermal treatment       \\
        Multimeter      & Fluke 115       & Fluke                 & Electrical measurements \\
        \bottomrule
    \end{tabular}
\end{table}

\subsection{Ingredients / Materials}
\begin{table}[H]
    \centering
    \caption{Materials and Chemicals}
    \begin{tabular}{@{}llll@{}}
        \toprule
        \textbf{Name} & \textbf{Purity / Grade}          & \textbf{Supplier} & \textbf{Role}    \\
        \midrule
        Ethanol       & 99.9\%                           & Sigma-Aldrich     & Solvent          \\
        Silicon Wafer & p-type, \textless100\textgreater & Wafer World       & Substrate        \\
        PEDOT:PSS     & 1.3 wt\% dispersion              & Heraeus           & Conductive layer \\
        \bottomrule
    \end{tabular}
\end{table}

\section{Preparation}
Describe any preparatory steps, such as:
\begin{itemize}
    \item Cleaning of substrates using solvents
    \item Preheating or calibration of equipment
    \item Solution preparation or mixing ratios
\end{itemize}

\section{Process / Experimental Procedure}
Outline each step of the main experimental process. Include:
\begin{itemize}
    \item Temperatures and durations
    \item Concentrations and volumes
    \item Equipment settings and sequences
\end{itemize}

\section{Characterisation Techniques}
\subsection*{Scanning Electron Microscopy (SEM)}
\textbf{Instrument:} Zeiss Sigma 300 \\
Used for surface morphology imaging. Operated at 5–10 kV.

\subsection*{X-ray Diffraction (XRD)}
\textbf{Instrument:} Bruker D8 Advance \\
Used to determine crystalline structure. Scanned from 10° to 80° in 2$\theta$.

\section{Results and Discussion}
Include figures, graphs, or tables of results:
\begin{figure}[H]
    \centering
    \includegraphics[width=0.6\textwidth]{example-image}
    \caption{Example result: surface morphology at 10kV SEM}
\end{figure}

Discuss trends, anomalies, and how the results align with expectations or literature.

\section{Conclusions}
Summarise:
\begin{itemize}
    \item Main findings
    \item Whether objectives were met
    \item Any limitations or sources of error
    \item Suggestions for improvement or further research
\end{itemize}

\section*{References}
\begin{itemize}
    \item J. Smith et al., \textit{Journal of Applied Physics}, 2022, 120(3), 1234.
    \item Equipment manuals and data sheets
    \item Scientific articles, standards
\end{itemize}

\appendix
\section*{Appendix}
Include raw data, code, calculations, or safety data sheets.

\end{document}